\documentclass[11pt]{article}

\usepackage{fullpage}

\begin{document}

\title{ARM Checkpoint Report}
\author{Sihan Tao, Taowen Liu, Ranchen Li ,Chuxuan Li}

\maketitle

\section{Group Organisation}

Taowen and Sihan first designed the structure of the emulator, and they specified the parameters and returning values of all required functions. Sihan designed intermediate representations of the all instructions. And he wrote the decoding part for 4 instructions. Then he wrote the dispatcher of execute and implemented mul instruction and some global helper functions. Taowen constructed the main loop also he is in charge of maintaining code quality and trouble shooting. He wrote the file loader and fetching part as well. Chuxuan wrote single data transfer instructions and Ranchen wrote branch and data processing instructions. Taowen joined in after he finished the main loop. While Chuxuan was working on tests for whole execution part, Taowen and Ranchen was solving bugs in execution part. Taowen, together with Ranchen and Chuxuan, refactored the whole execution part and extracted several common helper functions. And he organised the project structure and improved on data types definition and interfaces.\\

For each function, there were mainly two jobs to do: Writing the code and creating the unit test. After implementing the functions, each group member wrote a simple test program for these functions they have written. Sometimes we did other people's job to speed up the progress.\\

Our group works well together. We usually help each other through Teams video calls and Wechat whenever there are problems for writing codes or understanding the specification.\\

\section{Implementation Strategies}

We created some modules: Main, Decode, Execute, Utils and Tests.\\

Main module depends on Execute and Decode module. And everything requires Utils module since date types and common helpers are defined there.\\

Main module contains the \texttt{emulate.c} and its header. In \texttt{emulate.c}, there is a main function and several helpers (initializers and destructors). Fetch command is also defined in main module, because it is rather a short function. Main loop maintains the state of all registers, memory and all . We used old and new pipelines in order to avoid overlaying.\\

Decode module contains \texttt{decode.c} and its header. \texttt{decode.c} accepts a fetched instruction of unknown type as an argument and return the intermediate representation which defined in \texttt{types\_and\_macros.h}.\\

Execute module contains \texttt{execute\_helpers.c}, \texttt{execute\_DP.c}, \texttt{execute\_SDT.c} and \texttt{execute.c} as well as their headers. Data processing instruction is defined in \texttt{execute\_DP.c} and single data transfer instruction is defined in \texttt{execute\_SDT.c} since their logic are complicated. Multiplication and branch instruction is defined in \texttt{execute.c}. In \texttt{execute\_helpers.c}, there are common helpers (data rotation ...) since these helpers are commonly used in execution module, we put them into one file.\\

Utils module contains \texttt{tools.c}, \texttt{types\_and\_macros.h}, \texttt{unit\_test.c} and \texttt{file\_loader.c} and headers. \texttt{tools.c} has some bit operation and type transformation functions for general use. \texttt{types\_and\_macros.h} contains all the data types, structures and enumerations which are used in other functions. We defined all the instructions using a tag and a union of specific instructions. For specific instructions, we specified the exact length of parts in instructions such that they can fit in 32 bits. We also defined enum for opcodes.\\

Tests module contains all the module test source file.\\

\end{document}
